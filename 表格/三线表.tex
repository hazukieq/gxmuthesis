\documentclass[a4paper, UTF-8,14pt]{ctexart}
\usepackage{booktabs}
\usepackage{tabularx}
\usepackage{caption}

\newcommand{\tcap}[1]{
	\caption{\zihao{5}\kaishu #1}
}

\begin{document}
%前面的代码不用管,这里只是为了方便编译而写的
% 请直接将其复制到论文主体文件中
%为了快速正确生成表格,推荐使用以下工具:
%https://tableconvert.com/zh-cn/latex-generator
%https://www.tablesgenerator.com/

%经典三线表开始
\begin{table}[!h]
	\centering
	\tcap{动物研究} %表格题目
	%也可以使用 \caption{\zihao{5}\kaiti 表格题目}
	
	\label{tab:t1} %引用标签
	\begin{tabular}{cl}
		\toprule
		动物 & 详情 \\
		\midrule
		狗    & The dog is the most widely abundant terrestrial carnivore. \\
		\addlinespace
		猫    & The cat is a domestic species of small carnivorous mammal.\\
		\bottomrule
	\end{tabular}
\end{table}
%经典三线表结束

%[ht]表示 放在当前位置,放在顶部
%[!h]表示强制放在当前位置
\begin{table}[!h]
	\centering
	\tcap{动物研究} %表格题目
	%也可以使用 \caption{\zihao{5}\kaiti 表格题目}
	
	\label{tab:t2} %引用标签
	\begin{tabular}{cp{10cm}}
		\toprule
		动物 & 详情 \\
		\midrule
		狗    & The dog is the most widely abundant terrestrial carnivore. \\
		\addlinespace
		猫    & The cat is a domestic species of small carnivorous mammal.\\
		\bottomrule
	\end{tabular}
\end{table}
%经典三线表结束

%经典三线表
\begin{table}[!h]
	\centering
	\tcap{动物研究} %表格题目
	%也可以使用 \caption{\zihao{5}\kaiti 表格题目}
	\label{tab:t3} %引用标签
	
	%c表示居中,l表示左对齐,r表示右对齐
	%|表示竖线
	%故 {clr} 表示有三列,对齐方式分别是居中、左对齐、右对齐
	%指定列宽度,p{数字cm}/p{数字pt}/p{数字em}
	%故 {cp{6cm}p{4cm}},表示有三列,第一列是居中,第二列是指定宽度为6cm,第三列宽度为4cm
	\begin{tabular}{cp{10cm}}%两列,第一列居中第二列宽度为10cm
		\toprule
		动物 & 详情 \\
		\midrule
		狗    & The dog is the most widely abundant terrestrial carnivore. \\
		\addlinespace
		猫    & The cat is a domestic species of small carnivorous mammal.\\
		\bottomrule
	\end{tabular}
\end{table}

%经典表格,指定宽度
\begin{table}[!h]
	\centering
	\tcap{动物研究} %表格题目
	\begin{tabularx}{0.8\textwidth}
		{
			|>{\centering\arraybackslash}X %居中对齐
			|>{\raggedright\arraybackslash}X %左对齐
			|>{\raggedleft\arraybackslash}X| %右对齐
		}
		\hline
		动物 & 详情 & 属性 \\
		\hline
		狗    & The dog is the most widely abundant terrestrial carnivore. & 守护者\\
		\hline
		猫    & The cat is a domestic species of small carnivorous mammal. & 呵护者\\
		\hline
	\end{tabularx}
\end{table}

%经典三线表,指定列宽
\begin{table}[!h]
	\centering
	\tcap{动物研究} %表格题目
	%指定为一行文字宽度(\textwidth)的80%
	%或者指定为 6mm,6em,6pt等
	\begin{tabularx}{0.8\textwidth}
		{
			>{\raggedleft\arraybackslash}X %右对齐
			>{\centering\arraybackslash}X %居中
			>{\raggedright\arraybackslash}X %左对齐
		}
	\toprule
	动物 & 详情 & 属性 \\
	\midrule
	狗    & The dog is the most widely abundant terrestrial carnivore. & 守护者\\
	\addlinespace
	猫    & The cat is a domestic species of small carnivorous mammal. & 呵护者\\
	\bottomrule
	\end{tabularx}
\end{table}

\begin{figure}[htbp!]
	\centering
	\begin{minipage}[b]{.5\textwidth}
		\centering
		\captionof{table}{动物研究} %表格题目
		%也可以使用 \caption{\zihao{5}\kaiti 表格题目}
		\label{tab:t2} %引用标签
		\begin{tabular}{cp{6cm}}
			\toprule
			动物 & 详情 \\
			\midrule
			狗    & 厉害 \\
			\addlinespace
			猫    & 可爱\\
			\bottomrule
		\end{tabular}
	\end{minipage}
	\qquad
	\begin{minipage}[b]{.3\textwidth}
		\centering
		\captionof{table}{动物研究} %表格题目
		%也可以使用 \caption{\zihao{5}\kaiti 表格题目}
		
		\label{tab:t3} %引用标签
		\begin{tabular}{cp{3cm}}
			\toprule
			动物 & 详情 \\
			\midrule
			狗    & 厉害 \\
			\addlinespace
			猫    & 可爱\\
			\bottomrule
		\end{tabular}
	\end{minipage}
\end{figure}

\end{document}