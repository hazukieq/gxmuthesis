\documentclass[a4paper, UTF-8,14pt]{ctexart}
\usepackage{enumerate}
\usepackage[shortlabels]{enumitem}
\begin{document}
	\begin{itemize}
		\item 第一项
		\item 第二项
		\item 第三项
	\end{itemize}
	
	\section*{有序列表}
	\begin{enumerate}
		\item 第一项
		\item 第二项
		\item 第三项
	\end{enumerate}
	
	\section*{描述列表}
	\begin{description}
		\item[第一项] 这是第一项的描述。
		\item[第二项] 这是第二项的描述。
		\item[第三项] 这是第三项的描述。
	\end{description}
	
	\section*{嵌套列表}
	\begin{enumerate}
		\item 第一项
		\item 第二项
		\begin{itemize}
			\item 子项一
			\item 子项二
		\end{itemize}
		\item 第三项
	\end{enumerate}
	
	\section*{自定义无序列表}
	\begin{itemize}[label=\textbullet] % 使用圆点符号
		\item 第一项
		\item 第二项
		\item 第三项
	\end{itemize}
	
	
	\section*{其他类列表}
	\begin{enumerate}[label=(\Roman*)]
		\item 大写罗马字母+小括号1
		\item 大写罗马字母+小括号2 
	\end{enumerate}
	
	\begin{enumerate}[label=\heiti\chinese*、]
		\item 黑体中文编号一、
		\item 黑体中文编号二、
	\end{enumerate}
	
	\begin{enumerate}[label=\heiti\arabic*、]
		\item 黑体中文编号1、
		\item 黑体中文编号2、
	\end{enumerate}
	
	\begin{enumerate}[A.]
		\item 第一项内容
		\item 第二项内容
	\end{enumerate}
	
	\begin{enumerate}[\heiti a.]
		\item 第一项内容
		\item 第二项内容
	\end{enumerate}

	\section*{调整间距}
	\noindent itemsep 控制各项之间的垂直距离。\\
	topsep 控制列表与周围文本之间的距离。
	\begin{itemize}[itemsep=-0.4em,topsep=0pt,parsep=0pt]
		\item item1
		\item item2
		\item item3
	\end{itemize}
\end{document}