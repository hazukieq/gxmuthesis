%若想生成公式,请使用 AI 或者LaTex公式编辑器,并直接复制到论文主体中去
%https://www.latexlive.com/
%Axmath
%通过自然语言描述,让AI生成latex公式

%一个 \\(换行符) 会在当前行加上编号,不想要编号或者自定义编号的话
%请使用 \notag \\,\tag{自定义编号}\\
%单行公式,直接用 \notag,\tag{自定义编号},放在公式末尾

%行内公式书写
The Pythagorean theorem is $a^2 + b^2 = c^2$.

%段落式书写
The Pythagorean theorem is:
\begin{equation}
	a^2 + b^2 = c^2 \label{pythagorean}%这个是引用时的标签
	(4.1)
\end{equation}
%公式引用 \eqref{标签名}
Equation \eqref{pythagorean} is called ‘Gougu theorem’ in Chinese.


%多行公式对齐
\begin{align}
	a & = b + c \notag \\ %取消编号
	& = d + e
\end{align}

%多组公式对齐
\begin{align}
	a &=1 & b &=2 & c &=3\\
	d &=-1 & e &=-2 & f &=-5
\end{align}

%罗列多个公式,居中但不对齐
\begin{gather}
	a = b + c \\
	d = e + f + g \\
	h + i = j + k \notag \\
\end{gather}

%多个公式共用一个编号
\begin{equation}
	\begin{aligned}
		a &= b + c \\
		d &= e + f + g \\
		h &= j + k \\
		l &= m+n
	\end{aligned}
\end{equation}


% 不等式条件
%第一种写法
\begin{equation}
	|x| = \left\{
	\begin{array}{rl}
		-x & \text{if } x < 0,\\
		0 & \text{if } x = 0,\\
		x & \text{if } x > 0.
	\end{array} \right.
\end{equation}
%第二种写法
\begin{equation}
	|x| = \left\{
	\begin{cases}
		-x & \text{if } x < 0,\\
		0 & \text{if } x = 0,\\
		x & \text{if } x > 0.
	\end{cases}
\end{equation}

%证明
\begin{proof}
	为简单起见,我们使用 
	$$E=mc^2$$
\end{proof}