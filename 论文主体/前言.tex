%第一种,标题为「前言」,直接复制内容进去即可
%\front{前言章节的内容}

%第二种可用自定义标题
%空两格,请使用\hspace{1em}
\ftitle{前\hspace{1em}言}

我们用AI设计超越自然进化的超级蛋白,并通过AI蛋白质基础设施建设,把蛋白质设计这一科学问题转变为‘可编程’‘可预测’的工程问题。”许锦波表示,分子之心还将探索从蛋白质元件设计到代谢通路设计、工艺优化等生物经济领域研发、生产全流程的系统性优化设计。

谢诺投资创始合伙人、投委会主席魏晓林表示,从蛋白质结构预测到蛋白质从头设计是大的跨越,分子之心为此搭建的技术平台取得了不俗的进展,并在多个领域都有快速的商业化落地进展。这些商业化工作反过来又推动分子之心的技术平台加速迭代和丰富。

蛋白质是生物经济的核心物质基础,生物医药、生物制造等领域对功能性蛋白质需求巨大,但从自然界发现具备理想功能的蛋白质难度极大。传统方法耗时长,成本高,且成功率低,严重制约了生物制药、生物制造等领域的发展。

分子之心基于全球领先的AI蛋白质技术,融合分子动力学、量子化学等科学计算方法,实现小样本甚至零样本的蛋白质精准优化和设计,极大提升抗体、酶、疫苗等各类型蛋白质性能优化和从头设计的效率及成功率,在酶活性优化设计、极端环境下蛋白质性能优化等高难产业问题上取得突破。截至目前,分子之心已自研十余项性能领先的AI蛋白质预测、优化、设计算法,如产业级AI蛋白质生成大模型NewOrigin(达尔文)、可同时用于蛋白质侧链预测与序列设计的算法等,相关成果多次发表于SCIENCE杂志,获得首届国际基础科学大会前沿突破奖等荣誉。
