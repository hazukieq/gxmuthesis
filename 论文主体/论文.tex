\section{数里行间透视中国经济活力}
央视网消息:2025年前4个月相关经济数据出炉\cite{jin2001},创新驱动与产业集聚效应持续释放。此外,城镇老旧小区改造加快推进,城市更新行动也在加速中。\cite{tong1999}
\begin{pul}{2.5em}
	\item 第一项内容
	\item 第二项内容
	%\item 第一项内容 
\end{pul}

\subsection{前4个月国家高新区规上工业企业营收超10万亿元}
信息化部获悉,2025年前4个月,国家高新区规模以上工业企业营业
\pagebreak

\noindent 收入达10.26万亿元,同比增长7.3\%,创新驱动与产业集聚效应持续释放。\cite{riley1990}

\begin{pol}{\arabic*.}{3em}
	\item 第一项内容
	\item 第二项内容
\end{pol}

\begin{ol}
	\item 第一项内容
	\item 第二项内容
\end{ol}
其中,中关村新一代信息技术产业已跻身万亿级产业集群行列\cite{cctvnews},武汉东湖光电子信息的产业规模已占到全国的50\%。此外,国家高新区已布局量子信息、人形机器人、下一代互联网等前沿领域,相关未来产业已初步形成发展优势。\cite{han1996}

\begin{figure}[htbp!]
	\centering
	\begin{minipage}[b]{.5\textwidth}
		\centering
		\captionof{table}{动物研究} %表格题目
		%也可以使用 \caption{\zihao{5}\kaiti 表格题目}
		\label{tab:t2} %引用标签
		\begin{tabular}{cp{6cm}}
			\toprule
			动物 & 详情 \\
			\midrule
			狗    & 厉害 \\
			\addlinespace
			猫    & 可爱\\
			\bottomrule
		\end{tabular}
	\end{minipage}
	\qquad
	\begin{minipage}[b]{.3\textwidth}
		\centering
		\captionof{table}{动物研究} %表格题目
		%也可以使用 \caption{\zihao{5}\kaiti 表格题目}
		
		\label{tab:t3} %引用标签
		\begin{tabular}{cc}
			\toprule
			动物 & 详情 \\
			\midrule
			狗    & 厉害 \\
			\addlinespace
			猫    & 可爱\\
			\bottomrule
		\end{tabular}
	\end{minipage}
\end{figure}

%经典三线表
%为了快速正确生成表格,推荐使用以下工具:
%https://tableconvert.com/zh-cn/latex-generator
%https://www.tablesgenerator.com/
\begin{table}[ht]
	\centering
	\tcap{动物研究} %表格题目
	%也可以使用 \caption{\zihao{5}\kaiti 表格题目}
	
	\label{tab:t1} %引用标签
	\begin{tabular}{lp{10cm}}
		\toprule
		动物 & 详情 \\
		\midrule
		狗    & The dog is the most widely abundant terrestrial carnivore. \\
		\addlinespace
		猫    & The cat is a domestic species of small carnivorous mammal.\\
		\bottomrule
	\end{tabular}
\end{table}

\begin{table}[ht]
	\centering
	\tcap{动物研究} %表格题目
	\begin{tabularx}{0.8\textwidth}
		{
			|>{\centering\arraybackslash}X %居中对齐
			|>{\raggedright\arraybackslash}X %左对齐
			|>{\raggedleft\arraybackslash}X| %右对齐
		}
		\hline
		动物 & 详情 & 属性 \\
		\hline
		狗    & The dog is the most widely abundant terrestrial carnivore. & 守护者\\
		\hline
		猫    & The cat is a domestic species of small carnivorous mammal. & 呵护者\\
		\hline
	\end{tabularx}
\end{table}

\subsection{我国城镇老旧小区改造加快推进}
2025年以来,我国老旧小区改造加快推进,改造精细化水平不断提高。住房城乡建设部最新数据显示,2025年,全国计划新开工改造城镇老旧小区2.5万个。

\textbf{勾股定理}
\begin{equation}
	a^2 + b^2 = c^2 \label{pythagorean}
\end{equation}
方程 \eqref{pythagorean} 在英文中被称作 Pythagorean theorem。

\begin{align}
	a & = b + c \notag \\ %取消编号
	& = d + e
\end{align}

\begin{align}
	a &=1 & b &=2 & c &=3\\
	d &=-1 & e &=-2 & f &=-5
\end{align}

\begin{gather}
	a = b + c \\
	d = e + f + g \\
	h + i = j + k \notag \\
	l + m = n
\end{gather}

\begin{equation}
	\begin{aligned}
		a &= b + c \\
		d &= e + f + g \\
		h&= j + k \\
		l&= m+n
	\end{aligned}
\end{equation}

\begin{equation}
|x| = \left\{
\begin{array}{rl}
	-x & \text{if } x < 0,\\
	0 & \text{if } x = 0,\\
	x & \text{if } x > 0.
\end{array} \right.
\end{equation}

\begin{proof}
	为简单起见,我们使用 
	$$E=mc^2$$
\end{proof}

1--4月份,全国新开工改造城镇老旧小区5679个。河北、重庆、辽宁、上海、浙江、湖北等6个省市开工率超过50\%。河北、甘肃等地,聚焦环境脏乱等痛点,出台一系列措施协同推进治理。广东、山西等地通过引入智能设备、创新管理模式,营造便捷高效、安全有序的居住场景。宁夏、山东等地则积极为老旧小区增设各类服务设施,满足居民多元生活需求。

\subsection{中央财政支持北京等20城开展城市更新}
记者6月4日从财政部获悉,未来几年,财政部计划补助超过200亿元,支持北京、天津、唐山、包头、大连等20个城市实施城市更新行动,探索建立可持续的城市更新机制。
\begin{figure}[!ht]
	\centering
	\includegraphics[width=6cm]{图片/sample.jpg}
	\tcap{论文图片}
	\label{fig:f1}	
\end{figure}

\begin{figure}[!ht]
	\centering
	\begin{minipage}[b]{.4\linewidth}
		\centering
		\includegraphics[width=6cm]{图片/sample.jpg}
		\tcap{论文图片A}	
	\end{minipage}
	\qquad
	\begin{minipage}[b]{.4\linewidth}
		\centering
		\includegraphics[width=6cm]{图片/sample.jpg}
		\tcap{论文图片B}	
	\end{minipage}
\end{figure}

\begin{figure}[htbp!]
	\centering
	\includegraphics[width=6cm]{图片/sample.jpg}
	\qquad
	\includegraphics[width=6cm]{图片/sample.jpg}
	\tcap{论文并列图片}
\end{figure}

\subsubsection{中央财政下拨}
中央财政将按区域对实施城市更新行动的城市给予定额补助,每个城市补助不超过12亿元。资金将重点投向城市地下管网更新改造,生活垃圾分类处理等市政基础设施提升改造,历史文化街区等老旧片区更新改造等。推动补齐城市基础设施的短板弱项,促进城市基础设施建设由“有没有”向“好不好”转变\cite{caimin2006,Peebles2001-100-100,Alice13}。

\section{设计超级蛋白}
我们用AI设计超越自然进化的超级蛋白,并通过AI蛋白质基础设施建设,把蛋白质设计这一科学问题转变为‘可编程’‘可预测’的工程问题。”许锦波表示,分子之心还将探索从蛋白质元件设计到代谢通路设计\cite{jin1993}、工艺优化等生物经济领域研发、生产全流程的系统性``优化设计''。
\subsection{蛋白质}
\subsubsection{蛋白质结构}
\smallTiltle{2.1.1.1}{蛋白质结构预测}
谢诺投资创始合伙人、投委会主席魏晓林表示,从蛋白质结构预测到蛋白质从头设计是大的跨越,分子之心为此搭建的技术平台取得了不俗的进展,并在多个领域都有快速的商业化落地进展。这些商业化工作反过来又推动分子之心的技术平台加速迭代和丰富\cite{gill1985}。

%结语标题,这是可选的部分
\etitle{总结与展望}
%\section{总结与展望}%这样子写也是可以的,看个人

蛋白质是生物经济的核心物质基础,生物医药、生物制造等领域对功能性蛋白质需求巨大,但从自然界发现具备理想功能的蛋白质难度极大。传统方法耗时长,成本高,且成功率低,严重制约了生物制药、生物制造等领域的发展。\par
分子之心基于全球领先的AI蛋白质技术,融合分子动力学、量子化学等科学计算方法,实现小样本甚至零样本的蛋白质精准优化和设计,极大提升抗体、酶、疫苗等各类型蛋白质性能优化和从头设计的效率及成功率,在酶活性优化设计、极端环境下蛋白质性能优化等高难产业问题上取得突破。截至目前,分子之心已自研十余项性能领先的AI蛋白质预测、优化、设计算法,如产业级AI蛋白质生成大模型NewOrigin(达尔文)、可同时用于蛋白质侧链预测与序列设计的算法等,相关成果多次发表于SCIENCE杂志,获得首届国际基础科学大会前沿突破奖等荣誉。